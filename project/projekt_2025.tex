\documentclass[a4paper]{article}
\usepackage{hyperref}
\usepackage{amsmath}
\usepackage{fancyhdr}
\usepackage[top=10mm, bottom=10mm, left=10mm, right=10mm,includehead,includefoot]{geometry}


\pagestyle{empty}
\fancyhf{}

\lhead{LINKÖPINGS UNIVERSITET\\Avdelningen för statistik och maskininlärning\\Institutionen för datavetenskap}
\rhead{
Maskininlärning för statistiker 732G57 \\ HT2023}


\title{Projekt i Maskininlärning för statistiker 732G57}
\author{Josef Wilzén}



\begin{document}



\maketitle
\thispagestyle{fancy}


\section{Lärandemål}

Det huvudsakliga målet med denna inlämningsuppgift är att använda den teoretiska och 
praktiska kunskap som övats upp under första delen av kursen. Ni förväntas även få 
en praktisk övning i hur man kan analysera verkliga datamaterial samt de problem som 
kan uppstå med dessa. Det ingår även en övning i muntlig och skriftlig redovisning 
av analysresultatet. Instruktioner om vad som gäller kring generativ AI finns i sektion \ref{sec:genAI}.

\section{Instruktioner}

Er uppgift är att i par välja ett datamaterial som ni ska analysera. Se Sektion \ref{sec:data} för detaljer. När ni väl har valt ett datamaterial ska ni komma på en frågeställning som kan besvaras genom att analysera det valda datamaterialet. Det huvudsakliga problemet i projektet ska behandla övervakad inlärning. Det är dock tillåtet att använda oövervakad inlärning som del i att anlysera det huvudsakliga problemet. I analysen ska minst en modell vara ett neuralt nätverk.

Exempel på frågeställningar:
\begin{itemize}
    \item Vilka egenskaper påverkar hurvida en komponent är trasig? 
    \item Vilken modell ger bäst predikation av framtida inkomst?
    \item Vilken metod predikterar temperaturen bäst med avseende på MSE och MAE?
\end{itemize}
Under arbetets gång kommer ni säkert stöta på problem som till exempel att datamaterialet inte har det format som ni använt tidigare eller att en viss tilltänkt metod inte fungerar alls på det specifika datamaterialet. En del av denna inlämningsuppgift är att ni självständigt ska lösa dessa problem, men ni kan självklart ställa frågor under de schemalagda undervisningspassen. Lösningar som ni kommer på måste tydligt presenteras i rapporten som ni skriver för att uppfylla kravet om reproducerbarhet som råder för akademiska rapporter.

När ni väl kommit fram till ett svar på frågeställningen ska allting sammanställas i en rapport som ska formas enligt rapportmallen. Rapportmallen finns \href{https://raw.githubusercontent.com/STIMALiU/732G57_ML/master/project/template/Rapportmall STIMA projekt.rmd}{här} och \href{https://raw.githubusercontent.com/STIMALiU/732G57_ML/master/project/template/Rapportmall-STIMA-projekt.pdf}{här} (se även kurshemsidan), och innehåller instruktioner om hur ni ska skriva er rapport. Huvudfokus ligger på databeskrivningen och dess bearbetning samt rapportens metodkapitel. Alla analyser och slutsatser ska vara motiverade med lämpliga gradet och tabeller.

Rapporten ska skrivas med någon av följande programvaror:
\begin{itemize}
    \item \href{https://rmarkdown.rstudio.com/}{Rmarkdown}\footnote{Det går även att använda Quarto om man vill.} (med \href{https://yihui.org/knitr/}{knitr}) Rekomenderas för detta projekt!
    \item \href{https://en.wikipedia.org/wiki/LaTeX}{LaTeX} Vanligt vid skrivandet av akademiska rapporter. Kan använda \href{https://www.overleaf.com}{Overleaf} för att skriva tillsammans online (utan strul med installationer).
    \item \href{https://www.lyx.org/}{LyX} Grafiskt program som generar en LaTeX-rapport i bakgrunden. Kan användas med knitr för att köra R-kod.
\end{itemize}
Om ni använder knitr med Rmarkdown/latex/lyx, då rekomenderas det att ni har seprata R-filer där ni har era anayser, och att ni sen bara läser in lämpliga resultat i er rapport-fil. Så man inte måste köra om alla skattningar etc varje gånga som man ska kompliera sin rapport-fil. 

Rapporten ska lämnas in som en pdf-fil.  Döp filen på formen \texttt{gruppX\_liuid1\_liuid2.pdf} och ladda upp i rätt mapp på samarbetsytan på Lisam. Er rapport ska vara snygg och välstruktuerad. Se rapportmallen för mer instruktioner.


\subsection*{Tidigare studier}
En del datamaterial har analyserats av andra och olika analyser kan finnas tillgängliga på internet. Det är inte ok att direkt kopiera någon annans analys av samma datamaterial. Det är ok att använda vissa avgränsade delar av andras analyser om tydligt anger vad man använt och citerar källan. Exempel kan vara att man använder samma transformation av förklarande varaiabler som i en annan analys. Man kan hämta inspiration av vilka metoder som har använts eller inte använts på just ert datamaterial. Om ni läser om andra analyser av ert datamaterial så bör ni skriva kort om dessa i er bakgrundsektion i rapporten (och citera!), alltså vad som har gjorts tidigare på "området". 

Det är också ok att jämföra era egna resultat med resultatet från någon annans analys (glöm inte att citera då). Exempel: Ni har valt att använda neurala nätverk, ni ser att någon annan har använt random forest på samma data. Då kan ni i diskussionen citera den andra källan och jämföra era träffsäkerhet med deras träffsäkerhet. Det är inte ok att ta någon annas resultat från modellskattningar som ert eget resultat, utan ni måste skatta alla era egna modeller själv.

\subsection*{Datainlämning}
Ni ska göra en mindre inlämning på Lisam innan ni lämnar in den färdiga rapporten. Där ska ni:
\begin{itemize}
    \item Beskriva vilket datamaterial som ni har valt.
    \begin{itemize}
        \item vilka variabler, antal variabler, antal observationer osv.
        \item Vilken variabel är er responsvariabel? 
        \item kortfattad explorativ analys: kortfattad beskrivande statistik av data och/eller några plottar av data.
    \end{itemize}
    \item Ange preliminär frågeställning (ok att ändra senare vid behov).
\end{itemize}
Inlämningen ska vara en pdf-fil som är 1-3 sidor lång. Syftet är att ni ska välja data och komma igång med inledande datahantering, och börja fundera över frågeställningen. Det är ok att återanvända hela eller delar av denna inlämning till den slutgiltiga rapporten om man vill.

\subsection*{Presentation}
Under seminariet kommer det ges 25 minuter till varje grupp. Under de första 15 
minuterna ska ni presentera och sammanfatta den rapport som ni gjort, övriga 10 
minuter lämnas för opponering från opponentsgruppen. Alla gruppmedlemmar ska vara aktiva under presentationen.

Riktlinjer:
\begin{itemize}
    \item Ni ska beskriva ert problem och ert dataset tydligt. Förklara er datahantering.
    \item Metod: Ni behöver inte förklara all teori kring de metoder/modeller som ni har valt.
    \footnote{Historiskt så har det varit grupper som lagt ner mycket tid på att förklara teori kring metoder och inte hunnit prata om resutlat etc.}
    \begin{itemize}
        \item Undantag: Om ni har använt någon metod som ligger utanför det som 
        kursmaterialet har tagit upp, då behöver ni beskriva den metoden kortfattat så
        att andra som läser kursen kan förstå grunderna i metoden. Undvik att förklaringen tar för lång tid.
        \item Förklara kort vilka metoder/modeller som ni har valt. Förklara kort varför metoderna valdes. Förklara sen er
        praktiska metod och hur ni har tänkt kring modellering, hyperparametrar och utvärdering.
        \end{itemize}
    \item Presentera era resultat, diskussion och slutstaser. Se till att svara på er frågeställning.
\end{itemize}

Ladda upp era slides (som pdf) för presentationen på samarbetsytan i kursrummet innan seminariet.


\subsection*{Opponering}
Varje grupp ska opponera på en annan grupp. Det förväntas att fokus ligger på det statistika, det vill säga hur metoderna presenteras, används och tolkas.
\begin{itemize}
    \item Vid den muntliga opponeringen så ska de större konceptuella frågorna och kommentarerna tas upp.
    \begin{itemize}
      \item Det är inte ok bara säga något i stil med "allt var bra gjort". 
      Försök att (konstruktuvt) kritisera arbetet. Det är ok att kritisera delar som ni "håller med om". 
      Den muntliga opponeringen ska ta 7-10 min, så se till ha frågor/diskussion som fyller den tiden.
      \footnote{Om ni är osäkra på tiden så kan ni förbreda fler frågor än vad som kanske
      kommer att hinnas med, och så kör ni så många frågor som ni hinner under utsatt tid.} 
      \item Börja med de viktigaste/största frågorna när ni opponerar.
      \item Alla gruppmedlemmar ska vara aktiva under den muntliga opponeringen.
    \end{itemize}
    \item Mindre kommentarer och saker som rör formalia tas bara upp skriftligt.
\end{itemize}
Varje grupp ska sammanställa sina kommentarer i ett dokument som sedan ska skickas till rapportgruppen och lärare. Detta dokument ska innehålla både de små och stora kommentarerna.

\section{Datamaterial} \label{sec:data}
Varje grupp ska välja ett eget datamaterial. Två grupper kan inte ha samma datamaterial och först till kvarn gäller. På Lisam under sammarbetsyta finns ett excelark där ni kan skriva upp vilket datamaterial ni valt. Kom ihåg att citera källan på ert datamaterial i er rapport.

\subsection*{Välja datamaterial}
Ni är fria att välja ett eget datamaterial. Följande regler gäller:
\begin{itemize}
    \item Inget datamaterial som ni har arbetet med under datorlaborationerna under kursen.
    \item Inget simulerat datamaterial eller "toy data". Det ska vara ett riktigt datamaterial som kan användas för en riktig frågeställning.
    \item Inte för "enkelt": Tumregel, minst 1000 observationer eller minst 20 variabler. Fråga om ni är osäkra.
    \item När ni hittat ett datamaterial, skriv upp det i dokumentet. Fråga Josef om ni är osäkra på valet.
    \item Problemet ska vara inom \textbf{övervakad inlärning} (kan vara antingen regression eller klassificering).
    \item Ni får \textbf{inte} ta något datamaterial som användes på projektet föregående år, här finns en lista: \href{https://liuonline.sharepoint.com/:x:/r/sites/Lisam_732G57_2025HT_O2/CourseDocuments/datamaterial_projeket_2024.xlsx?d=wd84cf1abe387489aa32aa4cb91f34cca&csf=1&web=1&e=Xmv8rr}{länk}
\end{itemize}
Ni väljer själva var ni vill hämta data ifrån. Här kommer några förslag på datakällor:
\begin{itemize}
    \item \href{https://archive.ics.uci.edu/ml/index.php}{Machine Learning Repository}
    \item \href{https://www.kaggle.com/datasets}{Kaggle datasets}
    \item \href{https://medmnist.com/}{MedMNIST}
    \item \href{https://www.kdnuggets.com/datasets/index.html}{Datasets for Data Mining, Data Science, and Machine Learning}
    \item \href{https://github.com/awesomedata/awesome-public-datasets}{Awesome Public Datasets}
    \item \href{https://en.wikipedia.org/wiki/List_of_datasets_for_machine-learning_research}{List of datasets for machine-learning research}
    \item Kan också använda databaser från \href{https://cran.r-project.org/web/packages/pxweb/index.html}{pxweb}, se också \href{https://cran.r-project.org/web/packages/pxweb/vignettes/pxweb.html}{här} och \href{https://www.scb.se/en/services/statistical-programs-for-px-files/px-web/pxweb-examples/}{här}.
\end{itemize}



\section{Användning av generativ AI under projektet} \label{sec:genAI}

Det är tillåtet att använda  generativ AI för att underlätta er egen inlärning av 
kursens material, alltså använda det på ett sätt som ökar er kunskap och förståelse
för kursinnehållet samt ger stöd för ert lärande. Det går bra att bolla idéer och 
få feedback på sitt arbete med en generativ AI. 

Ni får inte ta text skriven av en generativ AI rakt av in i ert rapport. Om ni får
hjälp av en generativ AI på någon del i arbetet så behöver ni skriva om det med egna ord.
\newline
\newline
Det är inte tillåtet att:
\begin{itemize}
  \item Skriva hela eller delar av rapporten med generativ AI.
  \item Låta generativ AI ta fram all eller den mesta av den kod som används i projektet
  \item Ladda in er data direkt i en generativ AI och låta den skatta era modeller eller 
  ta fram era plottar/tabeller. Det är ok att få hjälp med kod som kan skapa t.ex. plottar.
  \item Låta generativ bestämma allt eller merparten av er ”praktiska metod”
  \footnote{För detaljer kring er praktiska metod, se rmd-mallen och dokumentet 'Projekt modellering' "} i projektet.
\end{itemize}


\subsection*{Om ni använder generativ AI}

Under Metod skriv ner vilka system för  generativ AI som ni använt när ni listar vilka programvaror ni använt i arbetet.
\newline
\newline
I bilagan finns en rubrik ”Generativ AI” här skriver ni mer detaljerat hur ni gått tillväga med att använda generativ AI i ert arbete, se nedan.
\newline
\newline
Om ni inte använt generativ AI så skriver ni bara något i stil med ”Generativ AI har inte används som hjälpmedel i detta arbete.” under rubriken ”Generativ AI” i bilagan.
\newline
\newline
\textbf{Förslag på struktur:}
\newline
Generellt: skriv mer allmänt hur ni använde generativ AI i arbetet
\newline
Sektion per sektion: hur använde ni generativ AI i varje given sektion?
\newline
\newline
Ni behöver inte redovisa varje prompt/anrop som ni gjort till en generativ AI, 
utan beskriv mer övergripande vad ni har gjort. Om ni använder kod från generativ AI,
som t.ex. är med i bilagan, ange då källan till koden.
\newline
\newline
\textbf{Notera:}  generativ AI kan generera svar som är falska eller inkorrekta på olika sätt. 
T.ex. så kan referenser skapas som inte finns i verkligheten. Kolla alltid upp 
källan om ni får tips på en referens av en AI. Ni har ansvar för att ert arbete
är korrekt och håller en bra kvalitet. Kolla upp i andra traditionella källor vid behov. Ni är ansvariga för det som står i ert arbete. 
\newline
\newline
Alla gruppdeltagare ska vara beredda att muntligen förklara alla delar av ert arbete, för att visa att ni har förståelse för det som ni har gjort.
\newline
\newline
Se till att använda generativ AI ansvarsfullt.

\end{document}