\documentclass[a4paper]{article}
\usepackage{hyperref}
\usepackage{amsmath}
\usepackage{fancyhdr}
\usepackage[top=10mm, bottom=10mm, left=10mm, right=10mm,includehead,includefoot]{geometry}


\pagestyle{empty}
\fancyhf{}

\lhead{LINKÖPINGS UNIVERSITET\\Avdelningen för statistik och maskininlärning\\Institutionen för datavetenskap}
\rhead{
Data Mining 732G12 \\ HT2023}


\title{Datum och deadlines för 732G12 HT2024}
\author{}
\date{}



\begin{document}



\maketitle
\thispagestyle{fancy}

\begin{itemize}
    \item Inlämning av datamaterial sker på Lisam, deadline \textbf{torsdag 26 september 18:00}.
    \begin{itemize}
        \item Skriv också upp ert valda datamaterial i excel-filen på sammarbetsytan.
        \item OBS! Inga grupper får jobba med samma datormaterial!
    \end{itemize}
    \item Tentan är 29 oktober kl 8 - 13 i datorsal (SU). (notera i början av HT2)
    \item Deadline för inläming av rapport för projektet är onsdag \textbf{16 oktober 16:00}.
    \begin{itemize}
        \item Presentation och opponeringslista kommer presenteras på kvällen på onsdagen. Rapporter som inte kommer med i tid kommer inte vara med i listan.
        \item Inlämning sker genom att ladda upp er pdf i korrekt folder under sammarbetsyta.
    \end{itemize}
    \item 17-18 oktober ska ni arbeta med er opponering och presentation. Börja med presentationen innan deadline!
    \item \textbf{21 oktober 9:15 - 17:00 är det seminarium} med presentation och opponering. Opponentgruppen ska maila sina kommentarer till respondenterna (och examinator) efter seminariet. \textbf{Deadline är 20:00 samma dag!}
    \item Söndagen den \textbf{27 oktober kl 20:00} är deadline för slutgiltiga rapporten.
    \begin{itemize}
        \item Ni ska komplettera er rapport utifrån de kommentarer ni får från opponenter och lärare.
        \item Denna inlämning sker i Lisam.
    \end{itemize}
\end{itemize}


\end{document}