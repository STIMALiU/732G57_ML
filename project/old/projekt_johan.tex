\documentclass[a4paper]{article}
\usepackage{hyperref}
\usepackage{amsmath}
\usepackage{fancyhdr}
\usepackage[top=10mm, bottom=10mm, left=10mm, right=10mm,includehead,includefoot]{geometry}


\pagestyle{empty}
\fancyhf{}

\lhead{LINKÖPINGS UNIVERSITET\\Avdelningen för statistik och maskininlärning\\Institutionen för datavetenskap}
\rhead{
Data Mining 732G12 \\ HT2023}


\title{Projekt i 732G12 Data Mining}
\author{Johan Alenlöv}



\begin{document}



\maketitle
\thispagestyle{fancy}


\section{Lärandemål}

Det huvudsakliga målet med denna inlämningsuppgift är att använda den teoretiska och praktiska kunskap som övats upp under första delen av kursen. Ni förväntas även få en praktisk övning i hur man kan analysera verkliga datamaterial samt de problem som kan uppstå med dessa. Det ingår även en övning i muntlig och skriftlig redovisning av analysresultatet.

\section{Instruktioner}

Er uppgift är att i par välja ett datamaterial som ni ska analysera. Se Sektion \ref{sec:data} för detaljer. När ni väl har valt ett datamaterial ska ni komma på en frågeställning som kan besvaras genom att analysera det valda datamaterialet. I analysen ska minst en modell vara ett neuralt nätverk.

Exempel på frågeställningar:
\begin{itemize}
    \item Vilka egenskaper påverkar hurvida en komponent är trasig? 
    \item Vilken modell ger bäst predikation av framtida inkomst?
    \item Vilken metod predikterar temperaturen bäst med avseende på MSE och MAE?
\end{itemize}
Under arbetets gång kommer ni säkert stöta på problem som till exempel att datamaterialet inte har det format som ni använt tidigare eller att en viss tilltänkt metod inte fungerar alls på det specifika datamaterialet. En del av denna inlämningsuppgift är att ni självständigt ska lösa dessa problem, men ni kan självklart ställa frågor under de schemalagda undervisningspassen. Lösningar som ni kommer på måste tydligt presenteras i rapporten som ni skriver för att uppfylla kravet om reproducerbarhet som råder för akademiska rapporter.

När ni väl kommit fram till ett svar på frågeställningen ska allting sammanställas i en rapport som ska formas enligt rapportmallen. Rapportmallen finns \href{https://raw.githubusercontent.com/STIMALiU/732G12_DM/master/project/template/Rapportmall STIMA projekt.rmd}{här} och \href{https://raw.githubusercontent.com/STIMALiU/732G12_DM/master/project/template/Rapportmall-STIMA-projekt.pdf}{här} (se även kurshemsidan), och innehåller instruktioner om hur ni ska skriva er rapport. Huvudfokus ligger på databeskrivningen och dess bearbetning samt rapportens metodkapitel. Alla analyser och slutsatser ska vara motiverade med lämpliga gradet och tabeller.

Rapporten ska skrivas med någon av följande programvaror:
\begin{itemize}
    \item \href{https://rmarkdown.rstudio.com/}{Rmarkdown} (med \href{https://yihui.org/knitr/}{knitr}) Rekomenderas för detta projekt!
    \item \href{https://en.wikipedia.org/wiki/LaTeX}{LaTeX} Vanligt vid skrivandet av akademiska rapporter. Kan använda \href{https://www.overleaf.com}{Overleaf} för att skriva tillsammans (utan strul med installationer).
    \item \href{https://www.lyx.org/}{LyX} Grafiskt program som generar en LaTeX-rapport i bakgrunden. Kan användas med knitr för att köra R-kod.
\end{itemize}
Rapporten ska lämnas in som en pdf-fil.  Döp filen på formen \texttt{gruppX\_liuid1\_liuid2.pdf} och ladda upp på Lisam.

\subsection*{Datainlämning}
Ni ska göra en mindre inlämning på Lisam innan ni lämnar in den färdiga rapporten. Där ska ni:
\begin{itemize}
    \item Beskriva vilket datamaterial som ni har valt.
    \begin{itemize}
        \item vilka variabler, antal variabler, antal observationer osv.
        \item kortfattad explorativ analys: kortfattad beskrivande statistik av data och/eller några plottar av data.
    \end{itemize}
    \item Ange preliminär frågeställning (ok att ändra senare vid behov).
\end{itemize}
Inlämningen ska vara en pdf-fil som är 1-3 sidor lång. Syftet är att ni ska välja data och komma igång med inledande datahantering, och börja fundera över frågeställningen. Det är ok att återanvända hela eller delar av denna inlämning till den slutgiltiga rapporten om man vill.

\subsection*{Presentation}
Under seminariet kommer det ges 25 minuter till varje grupp. Under de första 15 minuterna ska ni presentera och sammanfatta den rapport som ni gjort, övriga 10 minuter lämnas för opponering från opponentsgruppen.

\subsection*{Opponering}
Varje grupp ska opponera på en annan grupp. Det förväntas att fokus ligger på det statistika, det vill säga hur metoderna presenteras, används och tolkas.
\begin{itemize}
    \item Vid den muntliga opponeringen så ska de större konceptuella frågorna oh kommentarerna tas upp.
    \item Mindre kommentarer och saker som rör formalia tas bara upp skriftligt.
\end{itemize}
Varje grupp ska sammanställa sina kommentarer i ett dokument som sedan ska skickas till rapportgruppen och lärare. Detta dokument ska innehålla både de små och stora kommentarerna.

\section{Datamaterial} \label{sec:data}
Varje grupp ska välja ett eget datamaterial. Två grupper kan inte ha samma datamaterial och först till kvarn gäller. På Lisam under sammarbetsyta finns ett excel ark där ni kan skriva upp vilket datamaterial ni valt. Kom ihåg att citera källan på ert datamaterial i er rapport.

\subsection*{Välja datamaterial}
Ni är fria att välja ett eget datamaterial. Följande regler gäller:
\begin{itemize}
    \item Inget simulerat datamaterial eller "toy data". Det ska vara ett riktigt datamaterial som kan användas för en riktig frågeställning.
    \item Inte för "enkelt": Tumregel, minst 500 observationer eller minst 10 variabler. Fråga om ni är osäkra.
    \item När ni hittat ett datamaterial, skriv upp det i dokumentet. Fråga Johan om ni är osäkra på valet.
    \item Problemet kan vara antingen \textbf{regression} eller \textbf{klassificering}.
\end{itemize}
Här kan ni hitta data:
\begin{itemize}
    \item \href{https://archive.ics.uci.edu/ml/index.php}{Machine Learning Repository}
    \item \href{https://www.kaggle.com/datasets}{Kaggle datasets}
    \item \href{https://www.kdnuggets.com/datasets/index.html}{Datasets for Data Mining, Data Science, and Machine Learning}
    \item \href{https://en.wikipedia.org/wiki/List_of_datasets_for_machine-learning_research}{List of datasets for machine-learning research}
    \item Kan också använda databaser från \href{https://cran.r-project.org/web/packages/pxweb/index.html}{pxweb}, se också \href{https://cran.r-project.org/web/packages/pxweb/vignettes/pxweb.html}{här} och \href{https://www.scb.se/en/services/statistical-programs-for-px-files/px-web/pxweb-examples/}{här}.
\end{itemize}

\end{document}